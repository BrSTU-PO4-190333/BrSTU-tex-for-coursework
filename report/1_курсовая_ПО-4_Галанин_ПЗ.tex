\documentclass[12pt, a4paper, simple]{eskdtext}

\usepackage{hyperref}
\usepackage{_env/gpi_global.env}
\usepackage{_env/gpi_coursework.env}
\usepackage{_sty/gpi_lst}
\usepackage{_sty/gpi_toc}
\usepackage{_sty/gpi_t}
\usepackage{_sty/gpi_p}
\usepackage{_sty/gpi_u}

% Код
\ESKDletter{}{К}{Р}
\def \gpiDocTypeNum {81}
\def \gpiDocVer {00}
\def \gpiCode {\ESKDtheLetterI\ESKDtheLetterII\ESKDtheLetterIII.\gpiStudentGroupName\gpiStudentGroupNum.\gpiStudentCard-0\gpiDocNum~\gpiDocTypeNum~\gpiDocVer}

\def \gpiDocTopic {ПОЯСНИТЕЛЬНАЯ ЗАПИСКА}

% Графа 1 (наименование изделия/документа)
\ESKDcolumnI {\ESKDfontIII \gpiTopic \\ \gpiDocTopic}

% Графа 2 (обозначение документа)
\ESKDsignature {\gpiCode}

% Графа 9 (наименование или различительный индекс предприятия) задает команда
\ESKDcolumnIX {\gpiDepartment}

% Графа 11 (фамилии лиц, подписывающих документ) задают команды
\ESKDcolumnXIfI {\gpiStudentSurname}
\ESKDcolumnXIfII {\gpiTeacherSurname}
\ESKDcolumnXIfV {\gpiReviewerSurname}

\renewcommand {\thefigure} {\arabic{section}.\arabic{figure}}

\begin{document}
    \begin{ESKDtitlePage}
    \begin{center}
        \gpiMinEdu \\
        \gpiEdu \\
        \gpiKaf \\
    \end{center}

    \vfill

    \begin{center}
        \gpiTopic \\
    \end{center}

    \vfill

    \begin{center}
        \textbf{\gpiDocTopic} \\
        ПО ДИСЦИПЛИНЕ \gpiDiscipline \\
    \end{center}

    \vfill

    \begin{center}
        \gpiCode \\
        Листов \pageref{LastPage} \\
    \end{center}

    \vfill

    \begin{flushright}
        \begin{minipage}[t]{.49\textwidth}
            \begin{minipage}[t]{.75\textwidth}
                \begin{flushright}
                    Руководитель

                    Выполнил

                    Консультант

                    по ЕСПД
                \end{flushright}
            \end{minipage}
        \end{minipage}
        \begin{minipage}[t]{.49\textwidth}
            \begin{flushright}
                \begin{minipage}[t]{.75\textwidth}
                    \gpiTeacherName~\gpiTeacherSurname

                    \gpiStudentName~\gpiStudentSurname

                    \hspace{0pt}

                    \gpiTeacherName~\gpiTeacherSurname

                \end{minipage}
            \end{flushright}
            
        \end{minipage}
    \end{flushright}

    \vfill

    \begin{center}
        \ESKDtheYear
    \end{center}
\end{ESKDtitlePage}


    % Содержание
    \tableofcontents
    \thispagestyle{empty}                                
    \paragraph{ПРИЛОЖЕНИЕ А. СХЕМА ПРОГРАММЫ}
    \paragraph{ПРИЛОЖЕНИЕ Б. ТЕКСТ ПРОГРАММЫ}
    \newpage

    %
    \newpage
    \addcontentsline{toc}{section}{ВВЕДЕНИЕ}
    \section*{ВВЕДЕНИЕ}
    \newpage

    %
    \section{СИСТЕМНЫЙ АНАЛИЗ И ПОСТАНОВКА ЗАДАЧИ}
    \subsection{Перечень функций}
    \subsection{Требования пользователей}
    \newpage

    %
    \section{ПРОЕКТИРОВАНИЕ СИСТЕМЫ}
    \subsection{Проектирование архитектуры ПО (модули)}
    \subsection{Проектирование UI (макеты)}
    \newpage

    %
    \section{РЕАЛИЗАЦИЯ СИСТЕМЫ}
    \newpage

    %
    \section{ТЕСТИРОВАНИЕ СИСТЕМЫ}
    \newpage

    %
    \newpage
    \addcontentsline{toc}{section}{СПИСОК ИСПОЛЬЗОВАННЫХ ИСТОЧНИКОВ}
    \section*{СПИСОК ИСПОЛЬЗОВАННЫХ ИСТОЧНИКОВ}
    \begin{enumerate}
        \item[1.] ОСНОВЫ VAGRANT > Установка Vagrant на Windows 10 - [Электронный ресурс]
        Режим доступа: \url{https://www.youtube.com/watch?v=fESCSA-wQEQ}
        Дата~доступа:~13.02.2022.
        \item[2.] Установка Apache, PHP, MySQL (LAMP) на VDS сервер (в Ubuntu) - [Электронный ресурс]
        Режим доступа: \url{https://www.youtube.com/watch?v=FxwPQkP3OGY}
        Дата~доступа:~15.02.2022.
    \end{enumerate}
    \newpage
\end{document}
