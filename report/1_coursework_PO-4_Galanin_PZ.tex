\documentclass[
    12pt, % Устанавливает размер шрифта документа (12pt, 14pt)
    a4paper, % Лист А4
    simple, % Тоже самое что nocolumnsxix, nocolumnxxxi и nocolumnxxxii
    floatsection % Счётчик рисунков обнуляется каждую главу и нумеруется с номером главы, например, Рисунок 2.1
]{eskdtext}

\usepackage{hyperref}
\usepackage{_env/gpi_global.env}
\usepackage{_env/gpi_coursework.env}
\usepackage{_sty/gpi_lst}
\usepackage{_sty/gpi_toc}
\usepackage{_sty/gpi_t}
\usepackage{_sty/gpi_u}
\usepackage{_sty/gpi_list}
\usepackage{_sty/gpi_section}

% Код
\ESKDletter{}{К}{Р}
\def \gpiDocTypeNum {81}
\def \gpiDocVer {00}
\def \gpiCode {\ESKDtheLetterI\ESKDtheLetterII\ESKDtheLetterIII.\gpiStudentGroupName\gpiStudentGroupNum.\gpiStudentCard-0\gpiDocNum~\gpiDocTypeNum~\gpiDocVer}

\def \gpiDocTopic {ПОЯСНИТЕЛЬНАЯ ЗАПИСКА}

% Графа 1 (наименование изделия/документа)
\ESKDcolumnI {\ESKDfontIII \gpiTopic \\ \gpiDocTopic}

% Графа 2 (обозначение документа)
\ESKDsignature {\gpiCode}

% Графа 9 (наименование или различительный индекс предприятия) задает команда
\ESKDcolumnIX {\gpiDepartment}

% Графа 11 (фамилии лиц, подписывающих документ) задают команды

% Разраб.
\ESKDcolumnXIfI {\gpiStudentSurname}

% Пров.
\ESKDcolumnXIfII {\gpiTeacherSurname}

% (пустая строка после Пров. и перед Н. контр.)
% \ESKDcolumnXIfIV {Ббббб}

% Н. контр.
\ESKDcolumnXIfV {\gpiNormalControlSurname}

% Утв.
% \ESKDcolumnXIfVI {Ввввв}

\begin{document}
\begin{ESKDtitlePage}
    \begin{center}
        \gpiMinEdu \\
        \gpiEdu \\
        \gpiKaf \\
    \end{center}

    \vfill

    \begin{flushright}
        \begin{minipage}[t]{.45\textwidth}
            <<К защите допускаю>> \\
            \gpiHeadDepartmentInfo \\
            \underline{\hspace{3cm}} \gpiHeadDepartmentName~\gpiHeadDepartmentSurname \\
            \PageTitleDateField
        \end{minipage}
    \end{flushright}

    \vfill

    \begin{center}
        \gpiTopic \\
    \end{center}

    \vfill

    \begin{center}
        \textbf{\gpiDocTopic} \\
        ПО ДИСЦИПЛИНЕ \gpiDiscipline \\
    \end{center}

    \vfill

    \begin{center}
        \gpiCode \\
        Листов \pageref{LastPage} \\
    \end{center}

    \vfill

    \begin{flushright}
    \begin{minipage}[t]{.49\textwidth}
        \begin{minipage}[t]{.75\textwidth}
            \begin{flushright}
                Руководитель\\                          % Руководитель:
                \hspace{0pt}\\
                Выполнил\\                              % Выполнил
                \hspace{0pt}\\
                Консультанты:\\                         % Консультанты
                \gpiESPDInfo\\
                \gpiReviewerInfo\\
            \end{flushright}
        \end{minipage}
    \end{minipage}
    \begin{minipage}[t]{.49\textwidth}
        \begin{flushright}
            \begin{minipage}[t]{.75\textwidth}
                \gpiTeacherName~\gpiTeacherSurname\\    % Руководитель:
                \hspace{0pt}\\
                \gpiStudentName~\gpiStudentSurname\\    % Выполнил
                \hspace{0pt}\\
                \hspace{0pt}\\                          % Консультанты
                \gpiESPDName~\gpiESPDSurname\\
                \gpiReviewerName~\gpiReviewerSurname\\
            \end{minipage}
        \end{flushright}
    \end{minipage}
\end{flushright}


    \vfill

    \begin{center}
        \ESKDtheYear
    \end{center}
\end{ESKDtitlePage}



\ESKDthisStyle{empty}
Здесь лист с ТЗ.


% Содержание
\newpage
\ESKDthisStyle{formII}
\tableofcontents
\thispagestyle{empty} % кастыль, который уберает нумерацию по центру после содержания
\paragraph{ПРИЛОЖЕНИЕ А. СХЕМА ПРОГРАММЫ}
\paragraph{ПРИЛОЖЕНИЕ Б. ТЕКСТ ПРОГРАММЫ}


\newpage
\addcontentsline{toc}{section}{ВВЕДЕНИЕ}
\section*{ВВЕДЕНИЕ}

Далеко-далеко за словесными горами в стране гласных и согласных, живут рыбные тексты. На берегу снова что ipsum сбить журчит ручеек маленький продолжил семь эта. Грамматики, оксмокс, свой повстречался мир журчит вершину единственное запятой рукописи, переписывается текста власти жизни? Семь, рекламных языком страну силуэт, строчка однажды запятой знаках которой его возвращайся он. Безорфографичный эта своего, коварных над дороге, несколько однажды прямо но выйти продолжил, ты грамматики свой маленький себя. Все подпоясал рыбного вдали первую она семь алфавит ты предупредила инициал пунктуация. Ее дорогу всемогущая запятой языком залетают, все текста переулка использовало которой мир родного инициал пунктуация продолжил заманивший, маленькая грустный на берегу речью свой! Жаренные.

Далеко-далеко за словесными, горами в стране гласных и согласных живут рыбные тексты. Алфавит переулка свою сбить толку точках своих бросил встретил переписывается вершину жизни предложения послушавшись раз свое маленький о эта, букв домах даль родного путь своего реторический продолжил предупредила ручеек. Своих обеспечивает подпоясал на берегу, это сбить безорфографичный проектах он текст от всех первую моей единственное ручеек свое языкового которое предупредила до над рекламных встретил рот которой семь ему, агентство взобравшись! Которое домах переулка наш путь его гор рекламных последний, семантика на берегу необходимыми маленький. Мир, обеспечивает повстречался вдали но, рот переулка на берегу маленький родного ему которой составитель эта последний. Великий свое текстами правилами!

Далеко-далеко за словесными горами в стране гласных и согласных живут рыбные тексты. Ведущими собрал использовало переписывается пустился путь безорфографичный пояс буквоград она. Все речью текстов продолжил своих наш инициал заглавных маленькая заманивший переписали. Ему последний использовало агентство рот свой вдали маленький за своих правилами дал, решила жизни заголовок сих предложения вопрос она то жаренные запятых единственное мир. Оксмокс журчит большой осталось лучше предупреждал пор океана несколько все, пустился речью родного пунктуация переулка но встретил, безорфографичный своего страну, рукопись даль точках повстречался! Эта маленькая она, залетают, своих вскоре живет текстов, всеми имени предложения алфавит рот деревни lorem всемогущая возвращайся заманивший большого свой обеспечивает!


\newpage
\section{СИСТЕМНЫЙ АНАЛИЗ И ПОСТАНОВКА ЗАДАЧИ}

\subsection{Перечень функций}

Далеко-далеко за словесными, горами в стране гласных и согласных живут рыбные тексты. Текстов она меня проектах знаках коварных семь даже, заманивший послушавшись.

\begin{enumerate}
    \item Далеко-далеко за словесными, горами в стране гласных и согласных живут.
    \item Далеко-далеко за словесными, горами в стране гласных и согласных живут.
    \item Далеко-далеко за словесными, горами в стране гласных и согласных живут.
    \item Далеко-далеко за словесными, горами в стране гласных и согласных живут.
    \item Далеко-далеко за словесными, горами в стране гласных и согласных живут.
\end{enumerate}

\subsection{Требования пользователей}

\textbf{Пользовательские}:
\begin{enumerate}
    \item Далеко-далеко за словесными, горами в стране гласных и согласных живут.
    \item Далеко-далеко за словесными, горами в стране гласных и согласных живут.
    \item Далеко-далеко за словесными, горами в стране гласных и согласных живут.
    \item Далеко-далеко за словесными, горами в стране гласных и согласных живут.
    \item Далеко-далеко за словесными, горами в стране гласных и согласных живут.
\end{enumerate}

\textbf{Желательные}:
\begin{enumerate}
    \item Далеко-далеко за словесными, горами в стране гласных и согласных живут.
    \item Далеко-далеко за словесными, горами в стране гласных и согласных живут.
    \item Далеко-далеко за словесными, горами в стране гласных и согласных живут.
    \item Далеко-далеко за словесными, горами в стране гласных и согласных живут.
    \item Далеко-далеко за словесными, горами в стране гласных и согласных живут.
\end{enumerate}


\newpage
\section{ПРОЕКТИРОВАНИЕ СИСТЕМЫ}

\subsection{Проектирование архитектуры ПО (модули)}

\begin{figure}[!h]
    \centering
    % \includegraphics[]
    % {_assets/Flowcharts.png}
    \caption{UX}
\end{figure}

\subsection{Проектирование UI (макеты)}

\begin{figure}[!h]
    \centering
    % \includegraphics[]
    % {_assets/Flowcharts.png}
    \caption{UX}
\end{figure}

\begin{figure}[!h]
    \centering
    % \includegraphics[]
    % {_assets/Flowcharts.png}
    \caption{UI}
\end{figure}

\begin{table}[h!]
    \centering
    \begin{tabular}{|c|c|c|c|} 
        \hline
        Col1 & Col2 & Col2 & Col3 \\
        \hline
        1 & 2 & 3 & 4 \\ \hline
        5 & 6 & 7 & 8 \\ \hline
        1 & 2 & 3 & 4 \\ \hline
        5 & 6 & 7 & 8 \\ \hline
    \end{tabular}
    \caption{Table to test captions and labels}
    \label{table:1}
\end{table}

\newpage
\section{РЕАЛИЗАЦИЯ СИСТЕМЫ}

\begin{lstlisting}[name=Справочник "Производители"]
CREATE TABLE "catalog__producer" (
    ProducerCode integer NOT NULL PRIMARY KEY AUTOINCREMENT UNIQUE,
    ProducerName text
);
\end{lstlisting}

\begin{lstlisting}[name=Справочник "Единицы хранения"]
CREATE TABLE "catalog__unit_measure" (
    UnitMeasureCode integer NOT NULL PRIMARY KEY AUTOINCREMENT UNIQUE,
    UnitMeasureName text
);
\end{lstlisting}

\begin{lstlisting}[name=Справочник "Номенклатура"]
CREATE TABLE "catalog__nomenclature" (
    NomenclatureCode integer NOT NULL PRIMARY KEY AUTOINCREMENT UNIQUE,
    NomenclatureName text,
    ProducerCode integer,
    UnitMeasureCode integer,
    ReleaseYear integer
);
\end{lstlisting}

\newpage
\section{ТЕСТИРОВАНИЕ СИСТЕМЫ}

\begin{figure}[!h]
    \centering
    % \includegraphics[]
    % {_assets/Flowcharts.png}
    \caption{Результат вывода таблицы}
\end{figure}

\begin{figure}[!h]
    \centering
    % \includegraphics[]
    % {_assets/Flowcharts.png}
    \caption{Результат удаления таблицы}
\end{figure}

\begin{figure}[!h]
    \centering
    % \includegraphics[]
    % {_assets/Flowcharts.png}
    \caption{Результат добавления данных в таблицу}
\end{figure}

\begin{figure}[!h]
    \centering
    % \includegraphics[]
    % {_assets/Flowcharts.png}
    \caption{Результат обновления данных в таблице}
\end{figure}


\newpage
\addcontentsline{toc}{section}{СПИСОК ИСПОЛЬЗОВАННЫХ ИСТОЧНИКОВ}
\section*{СПИСОК ИСПОЛЬЗОВАННЫХ ИСТОЧНИКОВ}
\begin{enumerate}
    \item[1.] Коллекция eskdx v0.98 - eskdx.pdf
    [Электронный ресурс].
    Режим доступа: \url{http://tug.ctan.org/macros/latex/contrib/eskdx/manual/eskdx.pdf}.
    Дата доступа: 30.05.2022.

    \item[2.] Использование системы верстки LaTeX - EVMiS\_Latex.pdf
    [Электронный ресурс].
    Режим доступа: \url{https://www.bstu.by/uploads/attachments/metodichki/kafedri/EVMiS_Latex.pdf}.
    Дата доступа: 30.05.2022.

    \item[3.] Опции пакета hyperref
    [Электронный ресурс].
    Режим доступа: \url{https://grammarware.net/text/syutkin/hyperref_options.pdf}.
    Дата~доступа:~20.02.2022.

    \item[4.] Developers - Docker
    [Electronic resource].
    Mode of access: \url{https://www.docker.com/get-started/}.
    Date~of~access:~04.06.2022.

    \item[5.] Manual installation steps for older versions of WSL | Microsoft Docs
    [Electronic resource].
    Mode of access: \url{https://aka.ms/wsl2kernel}.
    Date~of~access:~04.06.2022.

    \item[6.] LaTeX/Source Code Listings - Wikibooks, open books for an open world
    [Electronic resource].
    Mode of access: \url{https://en.wikibooks.org/wiki/LaTeX/Source_Code_Listings}.
    Date~of~access:~04.06.2022.
\end{enumerate}


\newpage
\end{document}
