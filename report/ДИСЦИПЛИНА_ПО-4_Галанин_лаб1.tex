\documentclass[12pt, a4paper, simple]{eskdtext}

\usepackage{hyperref}
\usepackage{_env/gpi_global.env}
\usepackage{_env/gpi_report.env}
\usepackage{_sty/gpi_lst}
\usepackage{_sty/gpi_toc}
\usepackage{_sty/gpi_t}
\usepackage{_sty/gpi_p}
\usepackage{_sty/gpi_u}

% Код
% \ESKDletter{О}{Л}{Р}
% \def \gpiDocTypeNum {81}
% \def \gpiDocVer {00}
% \def \gpiCode {\ESKDtheLetterI\ESKDtheLetterII\ESKDtheLetterIII.\gpiStudentGroupName\gpiStudentGroupNum.\gpiStudentCard-0\gpiDocNum~\gpiDocTypeNum~\gpiDocVer}

\def \gpiDocTopic {Отчёт лабораторной работы №\gpiDocNum}

% Графа 1 (наименование изделия/документа)
% \ESKDcolumnI {\ESKDfontII \gpiTopic \\ \gpiDocTopic}

% Графа 2 (обозначение документа)
% \ESKDsignature {\gpiCode}

% Графа 9 (наименование или различительный индекс предприятия) задает команда
% \ESKDcolumnIX {\gpiDepartment}

% Графа 11 (фамилии лиц, подписывающих документ) задают команды
% \ESKDcolumnXIfI {\gpiStudentSurname}
% \ESKDcolumnXIfII {\gpiTeacherSurname}
% \ESKDcolumnXIfV {\gpiTeacherSurname}

\begin{document}
    \begin{ESKDtitlePage}
    \ESKDstyle{empty}
    \begin{center}
        \gpiMinEdu \\
        \gpiEdu \\
        \gpiKaf \\
    \end{center}

    \vfill

    \begin{center}
        \gpiTopic
    \end{center}

    \vfill

    \begin{center}
        \textbf{\gpiDocTopic} \\
        ПО ДИСЦИПЛИНЕ \gpiDiscipline \\
    \end{center}

    \vfill

    \begin{flushright}
        \begin{minipage}[t]{7cm}
            Выполнил:\\
            \PageTitleStudentInfo
            \PageTitleDateField
            \hspace{0pt}

            Проверил:\\
            \PageTitleTeacherInfo
            \PageTitleDateField
        \end{minipage}
    \end{flushright}

    \vfill

    \begin{center}
        \PageTitleCity~\ESKDtheYear
    \end{center}
\end{ESKDtitlePage}

    \ESKDstyle{empty}
    \begin{center}
        \textbf{\gpiDocTopic}
    \end{center}

    % = = = = = = = =
    \paragraph{} \textbf{Тема}: <<\gpiTopicRep>>

    \paragraph{} \textbf{Цель}: Сгенерировать отчёт, используя \LaTeX.

    \paragraph{} \textbf{Что нужно сделать}:

    \paragraph{} \textbf{Разработка дизайна}:

    % \begin{figure}[!h]
    %     \centering
    %     \includegraphics[]
    %         {_assets/ClassDiagram.png}
    %     \caption{Диграмма классов}
    % \end{figure}

    \paragraph{} \textbf{Исходный код}: 

    \lstinputlisting[language=c]{src/main.c}
    \lstinputlisting[language=java]{src/main.java}

    \begin{lstlisting}[caption=Вывод в консоль]
 Hello, World!
\end{lstlisting}

    \paragraph{} \textbf{Вывод}: создали отчёт, используя \LaTeX.

    % = = = = = = = =
    % \newpage
    % \addcontentsline{toc}{section}{Список использованных источников}
    \section*{Список использованных источников}
    \begin{enumerate}
        \item[1.] eskdx.pdf [Электронный ресурс]
        - Режим доступа: \url{http://tug.ctan.org/macros/latex/contrib/eskdx/manual/eskdx.pdf}.
        Дата~доступа:~20.02.2022.
        \item[2.] Опции пакета hyperref [Электронный ресурс]
        - Режим доступа: \url{https://grammarware.net/text/syutkin/hyperref_options.pdf}.
        Дата~доступа:~20.02.2022.
    \end{enumerate}
    \newpage
\end{document}
