\documentclass[
  12pt, % Устанавливает размер шрифта документа (12pt, 14pt)
  a4paper, % Лист А4
  simple, % Тоже самое что nocolumnsxix, nocolumnxxxi и nocolumnxxxii
  floatsection % Счётчик рисунков обнуляется каждую главу и нумеруется с номером главы, например, Рисунок 2.1
]{eskdtext}

\usepackage{env.config}
\usepackage{env.diplom.config}
\usepackage{_sty/sty@listing}
\usepackage{_sty/sty@lists}
\usepackage{_sty/sty@SectionMargins}
\usepackage{_sty/sty@table}
\usepackage{_sty/sty@TableOfContent}
\usepackage{_sty/sty@url}

\def \envDiplomDocumentTitle {ПОЯСНИТЕЛЬНАЯ ЗАПИСКА К ДИПЛОМНОМУ ПРОЕКТУ}
\def \envDiplomCode {\ESKDtheLetterI\ESKDtheLetterII\ESKDtheLetterIII.\envDiplomStudentGroupName\envDiplomStudentGroupNumber.\envDiplomStudentCard-0\envDiplomNumber~81~00}

\ESKDletter{}{Д}{П}
\ESKDcolumnI {\ESKDfontIII \envDiplomTitle \\ \envDiplomDocumentTitle} % наименование изделия/документа
\ESKDsignature {\envDiplomCode} % обозначение документа
\ESKDcolumnIX {\envDiplomTinyEnterprise} % наименование или различительный индекс предприятия
\ESKDcolumnXIfI {\envDiplomStudentSurname} % Разраб.
\ESKDcolumnXIfII {\envDiplomTeacherSurname} % Пров.
\ESKDcolumnXIfIV {} % (пустая строка после Пров. и перед Н. контр.)
\ESKDcolumnXIfV {\envDiplomNormControlSurname} % Н. контр.
\ESKDcolumnXIfVI {} % Утв.


\begin{document}
\begin{ESKDtitlePage}
    \begin{center}
        \envDiplomMinistr \\
        \envDiplomEducation \\
        \envDiplomUniversity \\
        \envDiplomCathedra \\
    \end{center}

    \vfill

    \begin{flushright}
        \begin{minipage}[t]{.45\textwidth}
            <<К защите допускаю>> \\
            \envDiplomHeadDepartmentInfo \\
            \underline{\hspace{3cm}} \envDiplomHeadDepartmentInitials~\envDiplomHeadCathedraSurname \\
            \envDiplomDateInput
        \end{minipage}
    \end{flushright}

    \vfill

    \begin{center}
        \envDiplomTitle \\
    \end{center}

    \vfill

    \begin{center}
        \textbf{\envDiplomDocumentTitle}
    \end{center}

    \vfill

    \begin{center}
        \envDiplomCode \\
        Листов \pageref{LastPage} \\
    \end{center}

    \vfill

    \begin{flushright}
    \begin{minipage}[t]{.49\textwidth}
        \begin{minipage}[t]{.75\textwidth}
            \begin{flushright}
                Руководитель\\                          % Руководитель:
                \hspace{0pt}\\
                Выполнил\\                              % Выполнил
                \hspace{0pt}\\
                Консультанты:\\                         % Консультанты
                \envDiplomEconomicConsultantInfo\\
                \envDiplomEspdConsultantInfo\\
                \envDiplomReviewerInfo\\
            \end{flushright}
        \end{minipage}
    \end{minipage}
    \begin{minipage}[t]{.49\textwidth}
        \begin{flushright}
            \begin{minipage}[t]{.75\textwidth}
                \envDiplomTeacherInitials~\envDiplomTeacherSurname\\    % Руководитель:
                \hspace{0pt}\\
                \envDiplomStudentInitials~\envDiplomStudentSurname\\    % Выполнил
                \hspace{0pt}\\
                \hspace{0pt}\\                          % Консультанты
                \envDiplomEconomicConsultantInitials~\envDiplomEconomicConsultantSurname\\
                \envDiplomEspdConsultantInitials~\envDiplomEspdConsultantSurname\\
                \envDiplomReviewerInitials~\envDiplomReviewerSurname\\
            \end{minipage}
        \end{flushright}
    \end{minipage}
\end{flushright}


    \vfill

    \begin{center}
        \ESKDtheYear
    \end{center}
\end{ESKDtitlePage}



\ESKDthisStyle{empty}
Здесь лист с ТЗ.


% Содержание
\newpage
\ESKDthisStyle{formII}
\tableofcontents
\thispagestyle{empty} % кастыль, который уберает нумерацию по центру после содержания
\paragraph{ПРИЛОЖЕНИЕ А. СХЕМА ПРОГРАММЫ}
\paragraph{ПРИЛОЖЕНИЕ Б. ТЕКСТ ПРОГРАММЫ}


\newpage
\addcontentsline{toc}{section}{ВВЕДЕНИЕ}
\section*{ВВЕДЕНИЕ}

...

\newpage
\section{СИСТЕМНЫЙ АНАЛИЗ И ПОСТАНОВКА ЗАДАЧИ}

\subsection{Перечень функций}

\textbf {Что-то}:
\begin{enumerate}
    \item один ;
    \item два .
\end{enumerate}

\subsection{Требования пользователей}

\textbf{Пользовательские требования пользователей}:
\begin{enumerate}
    \item один ;
    \item два .
\end{enumerate}

\textbf{Желательные требования пользователей}:
\begin{enumerate}
    \item один ;
    \item два .
\end{enumerate}


\newpage
\section{ПРОЕКТИРОВАНИЕ СИСТЕМЫ}

\subsection{Проектирование архитектуры ПО (модули)}

\begin{figure}[!h]
    \centering
    % \includegraphics[]
    % {_assets/Flowcharts.png}
    \caption{UX}
\end{figure}

\subsection{Проектирование UI (макеты)}

\begin{figure}[!h]
    \centering
    % \includegraphics[]
    % {_assets/Flowcharts.png}
    \caption{UX}
\end{figure}

\begin{figure}[!h]
    \centering
    % \includegraphics[]
    % {_assets/Flowcharts.png}
    \caption{UI}
\end{figure}

\begin{table}[h!]
    \centering
    \begin{tabular}{|c|c|c|c|} 
        \hline
        Col1 & Col2 & Col2 & Col3 \\
        \hline
        1 & 2 & 3 & 4 \\ \hline
        5 & 6 & 7 & 8 \\ \hline
        1 & 2 & 3 & 4 \\ \hline
        5 & 6 & 7 & 8 \\ \hline
    \end{tabular}
    \caption{Table to test captions and labels}
    \label{table:1}
\end{table}

\newpage
\section{РЕАЛИЗАЦИЯ СИСТЕМЫ}

\begin{lstlisting}[name=Справочник "Единицы хранения"]
CREATE TABLE "СП_ЕдиницыХранения" (
    Код integer NOT NULL PRIMARY KEY AUTOINCREMENT UNIQUE,
    Наименование text
);
\end{lstlisting}

\newpage
\section{ТЕСТИРОВАНИЕ СИСТЕМЫ}

\begin{figure}[!h]
    \centering
    % \includegraphics[]
    % {_assets/Flowcharts.png}
    \caption{Результат вывода таблицы}
\end{figure}

\begin{figure}[!h]
    \centering
    % \includegraphics[]
    % {_assets/Flowcharts.png}
    \caption{Результат удаления таблицы}
\end{figure}

\begin{figure}[!h]
    \centering
    % \includegraphics[]
    % {_assets/Flowcharts.png}
    \caption{Результат добавления данных в таблицу}
\end{figure}

\begin{figure}[!h]
    \centering
    % \includegraphics[]
    % {_assets/Flowcharts.png}
    \caption{Результат обновления данных в таблице}
\end{figure}


\newpage
\addcontentsline{toc}{section}{СПИСОК ИСПОЛЬЗОВАННЫХ ИСТОЧНИКОВ}
\section*{СПИСОК ИСПОЛЬЗОВАННЫХ ИСТОЧНИКОВ}
\begin{enumerate}
    \item[1.] Коллекция eskdx v0.98 - eskdx.pdf
    [Электронный ресурс].
    Режим доступа: \url{http://tug.ctan.org/macros/latex/contrib/eskdx/manual/eskdx.pdf}.
    Дата доступа: 30.05.2022.

    \item[2.] Использование системы верстки LaTeX - EVMiS\_Latex.pdf
    [Электронный ресурс].
    Режим доступа: \url{https://www.bstu.by/uploads/attachments/metodichki/kafedri/EVMiS_Latex.pdf}.
    Дата доступа: 30.05.2022.

    \item[3.] Опции пакета hyperref
    [Электронный ресурс].
    Режим доступа: \url{https://grammarware.net/text/syutkin/hyperref_options.pdf}.
    Дата~доступа:~20.02.2022.

    \item[4.] Developers - Docker
    [Electronic resource].
    Mode of access: \url{https://www.docker.com/get-started/}.
    Date~of~access:~04.06.2022.

    \item[5.] Manual installation steps for older versions of WSL | Microsoft Docs
    [Electronic resource].
    Mode of access: \url{https://aka.ms/wsl2kernel}.
    Date~of~access:~04.06.2022.

    \item[6.] LaTeX/Source Code Listings - Wikibooks, open books for an open world
    [Electronic resource].
    Mode of access: \url{https://en.wikibooks.org/wiki/LaTeX/Source_Code_Listings}.
    Date~of~access:~04.06.2022.
\end{enumerate}


\newpage
\end{document}
