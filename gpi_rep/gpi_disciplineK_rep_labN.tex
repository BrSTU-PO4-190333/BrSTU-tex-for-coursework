\documentclass[12pt, a4paper, simple]{eskdtext}

\usepackage{hyperref}
\usepackage{env}
\usepackage{_sty/gpi_lst}
\usepackage{_sty/gpi_toc}
\usepackage{_sty/gpi_t}
\usepackage{_sty/gpi_p}
\usepackage{_sty/gpi_u}

% Код
% \ESKDletter{О}{Л}{Р}
% \def \gpiDocTypeNum {81}
% \def \gpiDocVer {00}
% \def \gpiCode {\ESKDtheLetterI\ESKDtheLetterII\ESKDtheLetterIII.\gpiStudentGroupName\gpiStudentGroupNum.\gpiStudentCard-0\gpiDocNum~\gpiDocTypeNum~\gpiDocVer}

\def \gpiDocTopic {ОТЧЁТ ЛАБОРАТОРНОЙ РАБОТЫ №\gpiDocNum}

% Графа 1 (наименование изделия/документа)
% \ESKDcolumnI {\ESKDfontII \gpiTopic \\ \gpiDocTopic}

% Графа 2 (обозначение документа)
% \ESKDsignature {\gpiCode}

% Графа 9 (наименование или различительный индекс предприятия) задает команда
% \ESKDcolumnIX {\gpiDepartment}

% Графа 11 (фамилии лиц, подписывающих документ) задают команды
% \ESKDcolumnXIfI {\gpiStudentSurname}
% \ESKDcolumnXIfII {\gpiTeacherSurname}
% \ESKDcolumnXIfV {\gpiTeacherSurname}

\begin{document}
    \begin{ESKDtitlePage}
    \ESKDstyle{empty}
    \begin{center}
        \gpiMinEdu \\
        \gpiEdu \\
        \gpiKaf \\
    \end{center}

    \vfill

    \begin{center}
        \gpiTopic
    \end{center}

    \vfill

    \begin{center}
        \textbf{\gpiDocTopic} \\
        ПО ДИСЦИПЛИНЕ \gpiDiscipline \\
    \end{center}

    \vfill

    \begin{flushright}
        \begin{minipage}[t]{7cm}
            Выполнил:\\
            \PageTitleStudentInfo
            \PageTitleDateField
            \hspace{0pt}

            Проверил:\\
            \PageTitleTeacherInfo
            \PageTitleDateField
        \end{minipage}
    \end{flushright}

    \vfill

    \begin{center}
        \PageTitleCity~\ESKDtheYear
    \end{center}
\end{ESKDtitlePage}

    \ESKDstyle{empty}
    %
    \section*{СПИСОК ИСПОЛЬЗОВАННЫХ ИСТОЧНИКОВ}
    \addcontentsline{toc}{section}{СПИСОК ИСПОЛЬЗОВАННЫХ ИСТОЧНИКОВ}
    \begin{enumerate}
        \item[1.] ОСНОВЫ VAGRANT > Установка Vagrant на Windows 10 - [Электронный ресурс]
        URL: \url{https://www.youtube.com/watch?v=fESCSA-wQEQ}
        (дата~обращения:~13.02.2022).
        \item[2.] Установка Apache, PHP, MySQL (LAMP) на VDS сервер (в Ubuntu) - [Электронный ресурс]
        URL: \url{https://www.youtube.com/watch?v=FxwPQkP3OGY}
        (дата~обращения:~15.02.2022).
    \end{enumerate}
    \newpage
\end{document}
